\documentclass[a4paper, landscape]{article}

% Package
\usepackage{multicol, minted}
\usepackage[left=1cm,right=1cm,top=1cm,right=1cm]{geometry}

% Code box macro
\newenvironment{cpp}
	{\VerbatimEnvironment \begin{minted}[frame=single, tabsize=2, breaklines]{cpp}}
	{\end{minted}}

% Cover
\title{\textbf{ACM ICPC Cheat Sheet}}
\author{Fairuzi10}
\date{}

\begin{document}
\setlength{\columnseprule}{0.4pt}
\begin{multicols}{5}
	\tableofcontents
\end{multicols}
\setlength{\columnseprule}{0pt}
	
\begin{multicols*}{2}
\maketitle

\section{STL Useful Tips}

\subsection{GNU PBDS}
\begin{cpp}
#include <ext/pb_ds/assoc_container.hpp>
using namespace __gnu_pbds;

// change null_type to int to make it a map<int, int>
typedef tree<int, null_type, less<int>, rb_tree_tag,
tree_order_statistics_node_update> ordered_set;
\end{cpp}

\section{Geometry}

\subsection{Geometry Library}
\begin{cpp}
struct PT { 
  double x, y; 
  PT() {}
  PT(double x, double y) : x(x), y(y) {}
  PT(const PT &p) : x(p.x), y(p.y)    {}
  PT operator + (const PT &p)  const { return PT(x+p.x, y+p.y); }
  PT operator - (const PT &p)  const { return PT(x-p.x, y-p.y); }
  PT operator * (double c)     const { return PT(x*c,   y*c  ); }
  PT operator / (double c)     const { return PT(x/c,   y/c  ); }
};

double dot(PT p, PT q)     { return p.x*q.x+p.y*q.y; }
double dist2(PT p, PT q)   { return dot(p-q,p-q); }
double cross(PT p, PT q)   { return p.x*q.y-p.y*q.x; }
ostream &operator<<(ostream &os, const PT &p) {
  os << "(" << p.x << "," << p.y << ")"; 
}

// rotate a point CCW or CW around the origin
PT RotateCCW90(PT p)   { return PT(-p.y,p.x); }
PT RotateCW90(PT p)    { return PT(p.y,-p.x); }
PT RotateCCW(PT p, double t) { 
  return PT(p.x*cos(t)-p.y*sin(t), p.x*sin(t)+p.y*cos(t)); 
}
// project point c onto line through a and b assuming a != b
PT ProjectPointLine(PT a, PT b, PT c) {
  return a + (b-a)*dot(c-a, b-a)/dot(b-a, b-a);
}

// project point c onto line segment through a and b
PT ProjectPointSegment(PT a, PT b, PT c) {
  double r = dot(b-a,b-a);
  if (fabs(r) < EPS) return a;
  r = dot(c-a, b-a)/r;
  if (r < 0) return a;
  if (r > 1) return b;
  return a + (b-a)*r;
}
// compute distance from c to segment between a and b
double DistancePointSegment(PT a, PT b, PT c) {
  return sqrt(dist2(c, ProjectPointSegment(a, b, c)));
}
// compute distance between point (x,y,z) and plane ax+by+cz=d
double DistancePointPlane(double x, double y, double z,
                          double a, double b, double c, double d)
{
  return fabs(a*x+b*y+c*z-d)/sqrt(a*a+b*b+c*c);
}
// determine if lines from a to b and c to d are parallel or collinear
bool LinesParallel(PT a, PT b, PT c, PT d) { 
  return fabs(cross(b-a, c-d)) < EPS; 
}
bool LinesCollinear(PT a, PT b, PT c, PT d) { 
  return LinesParallel(a, b, c, d)
      && fabs(cross(a-b, a-c)) < EPS
      && fabs(cross(c-d, c-a)) < EPS; 
}
// determine if line segment from a to b intersects with line segment from c to d
bool SegmentsIntersect(PT a, PT b, PT c, PT d) {
  if (LinesCollinear(a, b, c, d)) {
    if (dist2(a, c) < EPS || dist2(a, d) < EPS ||
      dist2(b, c) < EPS || dist2(b, d) < EPS) return true;
    if (dot(c-a, c-b) > 0 && dot(d-a, d-b) > 0 && dot(c-b, d-b) > 0)
      return false;
    return true;
  }
  if (cross(d-a, b-a) * cross(c-a, b-a) > 0) return false;
  if (cross(a-c, d-c) * cross(b-c, d-c) > 0) return false;
  return true;
}
// compute intersection of line passing through a and b with line passing through c and d, assuming that unique intersection exists; for segment intersection, check if segments intersect first
PT ComputeLineIntersection(PT a, PT b, PT c, PT d) {
  b=b-a; d=c-d; c=c-a;
  assert(dot(b, b) > EPS && dot(d, d) > EPS);
  return a + b*cross(c, d)/cross(b, d);
}
// compute circumcenter of circle given three points
PT ComputeCircleCenter(PT a, PT b, PT c) {
  b=(a+b)/2;
  c=(a+c)/2;
  return ComputeLineIntersection(b, b+RotateCW90(a-b), c, c+RotateCW90(a-c));
}
// compute incenter of circle given three points
PT ComputeInCenter(PT a, PT b, PT c) {
	double x = hypot(b.x-c.x,b.y-c.y);
	double y = hypot(a.x-c.x,a.y-c.y);
	double z = hypot(a.x-b.x,a.y-b.y);

	double rx = x*a.x+y*b.x+z*c.x;
	double ry = x*a.y+y*b.y+z*c.y;
	return PT(rx,ry)/(x+y+z);
}

// check ccw & count angle
bool ccw(PT p, PT q, PT r) { return cross(PT(p, q), PT(p, r)) > 0; }
double angle(PT a, PT o, PT b) { // return AOB in rad
	PT oa = PT(o, a), ob = PT(o, b);
	return acos(dot(oa, ob)/sqrt(dist2(oa, PT(0, 0))*dist2(ob, PT(0, 0))));
}
// test point in polygon from sum of angle
bool PointInPolygon(const vector<PT> &p, PT q) {
	double sum = 0;
	for (int i = 0; i < (int) p.size(); i++) {
		if (ccw(q, p[i], p[i+1])) 
			sum += angle(p[i], q, p[(i+1)%p.size()]);
		else 
			sum -= angle(p[i], q, p[(i+1)%p.size()]);
	}

	return fabs(fabs(sum) - 2*PI) < EPS;
}
// determine if point is on the boundary of a polygon
bool PointOnPolygon(const vector<PT> &p, PT q) {
  for (int i = 0; i < p.size(); i++)
    if (dist2(ProjectPointSegment(p[i], p[(i+1)%p.size()], q), q) < EPS)
      return true;
    return false;
}
// compute intersection of line through points a and b with circle centered at c with radius r > 0
vector<PT> CircleLineIntersection(PT a, PT b, PT c, double r) {
  vector<PT> ret;
  b = b-a;
  a = a-c;
  double A = dot(b, b);
  double B = dot(a, b);
  double C = dot(a, a) - r*r;
  double D = B*B - A*C;
  if (D < -EPS) return ret;
  ret.push_back(c+a+b*(-B+sqrt(D+EPS))/A);
  if (D > EPS)
    ret.push_back(c+a+b*(-B-sqrt(D))/A);
  return ret;
}
// compute intersection of circle centered at a with radius r with circle centered at b with radius R
vector<PT> CircleCircleIntersection(PT a, PT b, double r, double R) {
  vector<PT> ret;
  double d = sqrt(dist2(a, b));
  if (d > r+R || d+min(r, R) < max(r, R)) return ret;
  double x = (d*d-R*R+r*r)/(2*d);
  double y = sqrt(r*r-x*x);
  PT v = (b-a)/d;
  ret.push_back(a+v*x + RotateCCW90(v)*y);
  if (y > 0)
    ret.push_back(a+v*x - RotateCCW90(v)*y);
  return ret;
}
// This code computes the area or centroid of a (possibly nonconvex) polygon, assuming that the coordinates are listed in a clockwise or counterclockwise fashion. Note that the centroid is often known as the "center of gravity" or "center of mass".
double ComputeSignedArea(const vector<PT> &p) {
  double area = 0;
  for(int i = 0; i < p.size(); i++) {
    int j = (i+1) % p.size();
    area += p[i].x*p[j].y - p[j].x*p[i].y;
  }
  return area / 2.0;
}
PT ComputeCentroid(const vector<PT> &p) {
  PT c(0,0);
  double scale = 6.0 * ComputeSignedArea(p);
  for (int i = 0; i < p.size(); i++){
    int j = (i+1) % p.size();
    c = c + (p[i]+p[j])*(p[i].x*p[j].y - p[j].x*p[i].y);
  }
  return c / scale;
}
\end{cpp}

\section{Graph}

\subsection{Max Flow Dinic}
\begin{cpp}
// run in O(V^2*E)
bool bfs() {
	fill_n(lvl, MAXN, INF);
	lvl[SOURCE] = 0;
	q.push(SOURCE);
	while (!q.empty()) {
		int now = q.front();
		q.pop();
		if (lvl[now]+1 > lvl[SINK]) continue;
		for (auto i: edge[now]) {
			if (lvl[now]+1 < lvl[i] && rem[now][i]) {
				lvl[i] = lvl[now]+1;
				q.push(i); 
	}}}
	return lvl[SINK] != INF;
}
int dfs(int now, int cur_flow) {
	if (now == SINK) return cur_flow;
	int used_flow = 0;
	for (auto i: edge[now]) {
		if (lvl[i] == lvl[now]+1 && rem[now][i]) {
			int next_flow = dfs(i, min(rem[now][i], cur_flow-used_flow));
			used_flow += next_flow;
			rem[now][i] -= next_flow;
			rem[i][now] += next_flow;
			if (used_flow == cur_flow) return used_flow;
	}}
	return used_flow;
}

// in main()
while (bfs()) {
	ans += dfs(SOURCE, INF);
}
\end{cpp}

\subsection{Bipartite Matching Hopcroft Karp}
\begin{cpp}
// run in O(E*sqrt(V))
bool bfs() {
	memset(lvl, 63, sizeof(lvl));
	for (int i = 0; i < N; i++) {
		if (pairL[i] == NIL) {
			lvl[i] = 0;
			q.push(i); }}
	while (!q.empty()) {
		int now = q.front(); q.pop();
		for (auto i: edge[now]) {
			if (lvl[pairR[i]] > lvl[now]+1) {
				lvl[pairR[i]] = lvl[now]+1;
				q.push(pairR[i]); 
	}}}
	return lvl[NIL] < INF;
}

bool dfs(int now) {
	if (now == NIL) return 1;
	for (auto i: edge[now]) {
		if (lvl[pairR[i]] == lvl[now]+1) {
			if (dfs(pairR[i])) {
				pairL[now] = i;
				pairR[i] = now;
				return 1;
			} else lvl[pairR[i]] = INF; 
	}}
	return 0;
}

int bipartite_matching() {
	for (int i = 0; i < N; i++) pairL[i] = NIL;
	for (int i = 0; i < M; i++) pairR[i] = NIL;
	int ret = 0;
	while (bfs()) {
		for (int i = 0; i < N; i++) {
			if (lvl[i] == 0) {
				if (dfs(i)) ret++;
				else lvl[i] = INF;
	}}}
	return ret;
}
\end{cpp}

\section{String}

\subsection{Suffix Array}
\begin{cpp}
// change sa[i]+k to (sa[i]+k)%N if cyclic
// change sa[i]+k to min(sa[i]+k, sa[i]+limit-k) and k < N to k < limit if the substring length is limited.

void radix(int k) {
	int maxi = max(300, N);
	memset(cnt, 0, sizeof(cnt));
	for (int i = 0; i < N; i++) cnt[sa[i]+k < N? ra[sa[i]+k]: 0]++;
	for (int i = 0; i < maxi; i++) cnt[i] += cnt[i-1];
	for (int i = N-1; i >= 0; i--) tsa[--cnt[sa[i]+k < N? ra[sa[i]+k]: 0]] = sa[i];
	for (int i = 0; i < N; i++) sa[i] = tsa[i];
}

void compute_sa() {
	for (int i = 0; i < N; i++) ra[i] = str[i];
	for (int i = 0; i < N; i++) sa[i] = i;
	for (int k = 1; k < N; k <<= 1) {
		radix(k);
		radix(0);
		tra[sa[0]] = 0;
		for (int i = 1; i < N; i++)
			tra[sa[i]] = ra[sa[i]] == ra[sa[i-1]] && 
				ra[sa[i]+k] == ra[sa[i-1]+k]? tra[sa[i-1]]: tra[sa[i-1]]+1;
		for (int i = 0; i < N; i++) ra[i] = tra[i];
		if (ra[sa[N-1]] == N-1) break;
}}

void compute_lcp() {
	int cur_lcp = 0;
	for (int i = 0; i < N; i++) {
		if (ra[i] == 0) {
			lcp[i] = 0;
			continue;
		}
		cur_lcp = max(0, cur_lcp-1);
		while (str[i+cur_lcp] == str[sa[ra[i]-1]+cur_lcp]) cur_lcp++;
		lcp[i] = cur_lcp;
}}

// in main() 
str += '$' // if not cyclic
N = str.length()
\end{cpp}

\end{multicols*}
\end{document}