\section{Graph}

\subsection{Max Flow Dinic}
\begin{cppcode}
	// run in $O(V^2E)$
	bool bfs() {
		fill_n(lvl, MAXN, INF);
		lvl[SOURCE] = 0;
		q.push(SOURCE);
		while (!q.empty()) {
				int now = q.front();
				q.pop();
				if (lvl[now]+1 > lvl[SINK]) continue;
				for (auto i: edge[now]) {
						if (lvl[now]+1 < lvl[i] && rem[now][i]) {
								lvl[i] = lvl[now]+1;
								q.push(i);
		}}}
		return lvl[SINK] != INF;
	}
	int dfs(int now, int cur_flow) {
		if (now == SINK) return cur_flow;
		int used_flow = 0;
		for (auto i: edge[now]) {
				if (lvl[i] == lvl[now]+1 && rem[now][i]) {
						int next_flow = dfs(i, min(rem[now][i], cur_flow-used_flow));
						used_flow += next_flow;
						rem[now][i] -= next_flow;
						rem[i][now] += next_flow;
						if (used_flow == cur_flow) return used_flow;
		}}
		return used_flow;
	}

	// in main()
	while (bfs()) {
		ans += dfs(SOURCE, INF);
	}
\end{cppcode}

\subsection{Bipartite Matching Kuhn}
\begin{cppcode}
	// run in $O(EV)$
	bool kuhn(int now) {
		for (auto i: edge[now]) {
			if (sudah[i]) continue;
			sudah[i] = 1;
			if (par[i] == -1 || kuhn(par[i])) {
				par[i] = now;
				return 1;
			}
		}
		return 0;
	}
	
	// in main()
	for (int i = 0, n = row.size(); i < n; i++) {
		sudah.reset();
		if (kuhn(i)) ans++;
	}
\end{cppcode}

\subsection{Bipartite Matching Hopcroft Karp}
\begin{cppcode}
	// run in $O(E\sqrt{V})$
	bool bfs() {
		memset(lvl, 63, sizeof(lvl));
		for (int i = 0; i < N; i++) {
				if (pairL[i] == NIL) {
						lvl[i] = 0;
						q.push(i); }}
		while (!q.empty()) {
				int now = q.front(); q.pop();
				for (auto i: edge[now]) {
						if (lvl[pairR[i]] > lvl[now]+1) {
								lvl[pairR[i]] = lvl[now]+1;
								q.push(pairR[i]);
		}}}
		return lvl[NIL] < INF;
	}

	bool dfs(int now) {
		if (now == NIL) return 1;
		for (auto i: edge[now]) {
				if (lvl[pairR[i]] == lvl[now]+1) {
						if (dfs(pairR[i])) {
								pairL[now] = i;
								pairR[i] = now;
								return 1;
							} else lvl[pairR[i]] = INF;
		}}
		return 0;
	}

	int bipartite_matching() {
		for (int i = 0; i < N; i++) pairL[i] = NIL;
		for (int i = 0; i < M; i++) pairR[i] = NIL;
		int ret = 0;
		while (bfs()) {
				for (int i = 0; i < N; i++) {
						if (lvl[i] == 0) {
								if (dfs(i)) ret++;
								else lvl[i] = INF;
		}}}
		return ret;
	}
\end{cppcode}

\subsection{Min Cost Bipartite Matching}
\begin{cppcode}
	// make sure N <= M
	vector<int> u(N+1), v(M+1), p(M+1), way(M+1);
	for (int i=1; i<=N; ++i) {
		p[0] = i;
		int j0 = 0;
		vector<int> minv (M+1, INF);
		vector<char> used (M+1, false);
		do {
			used[j0] = true;
			int i0 = p[j0],  delta = INF,  j1;
			for (int j=1; j<=M; ++j) {
				if (!used[j]) {
					int cur = a[i0][j]-u[i0]-v[j];
					if (cur < minv[j]) minv[j] = cur,  way[j] = j0;
					if (minv[j] < delta) delta = minv[j],  j1 = j;
				}
			}
			for (int j=0; j<=M; ++j) {
				if (used[j]) u[p[j]] += delta,  v[j] -= delta;
				else minv[j] -= delta;
			}
			j0 = j1;
		} while (p[j0] != 0);
		do {
			int j1 = way[j0];
			p[j0] = p[j1];
			j0 = j1;
		} while (j0);
	}
	int cost = -v[0]

	// if you need to know pair <row, col>
	vector<int> ans (N+1);
	for (int j = 1; j <= M; j++) ans[p[j]] = j;
\end{cppcode}